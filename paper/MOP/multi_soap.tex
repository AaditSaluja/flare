\documentclass[%
% reprint,
%superscriptaddress,
%groupedaddress,
%unsortedaddress,
%runinaddress,
%frontmatterverbose, 
preprint,
%showpacs,preprintnumbers,
%nofootinbib,
%nobibnotes,
%bibnotes,
amsmath,amssymb,
aps,
%pra,
%prb,
%rmp,
%prstab,
%prstper,
%floatfix,
]{revtex4-1}

\usepackage{braket}
\usepackage{graphicx}% Include figure files
% \usepackage{dcolumn}% Align table columns on decimal point
\usepackage{bm}% bold math
%\usepackage{hyperref}% add hypertext capabilities
%\usepackage[mathlines]{lineno}% Enable numbering of text and display math
%\linenumbers\relax % Commence numbering lines

%\usepackage[showframe,%Uncomment any one of the following lines to test 
%%scale=0.7, marginratio={1:1, 2:3}, ignoreall,% default settings
%%text={7in,10in},centering,
%%margin=1.5in,
%%total={6.5in,8.75in}, top=1.2in, left=0.9in, includefoot,
%%height=10in,a5paper,hmargin={3cm,0.8in},
%]{geometry}

\begin{document}

\title{Multi-component kernel notes}

\author{Jonathan Vandermause}
% \affiliation{$^{1}$John A. Paulson School of Engineering and Applied
% Sciences, Harvard University, Cambridge, Massachusetts 02138, USA \\
% $^{2}$Department of Physics, Harvard University, Cambridge, Massachusetts
% 02138, USA \\
% $^{3}$Center for Computational Engineering, Massachusetts Institute of
% Technology, Cambridge, Massachusetts 02139, USA \\
% $^{4}$ Department of Mechanical Engineering, Massachusetts Institute of Technology, Cambridge, Massachusetts 02139, USA}

\date{\today}

% \begin{abstract}
% \end{abstract}

\maketitle

\section{Computing a covariant N-body kernel}

\section{SOAP is a three-body kernel}
Consider the multi-component SOAP kernel defined in Ref.\ \cite{de2016comparing},
\begin{equation}
    k(\rho_1, \rho_2) = \int d\hat{R} \left( \int d\vec{r} \sum_{\alpha \alpha'} \kappa_{\alpha \alpha'} \rho_1^\alpha(\vec{r}) \rho_2^{\alpha'} (\hat{R} \vec{r}) \right)^2.
\end{equation}
Recalling that
\begin{equation}
    \rho_1^\alpha(\vec{r}) = \sum_{i \in \rho_1^\alpha} \exp\left(-\frac{(\vec{x}_i - \vec{r})^2}{2\sigma^2}\right)
\end{equation}
and performing the integral gives
\begin{equation}
\int d\vec{r} \sum_{\alpha \alpha'} \kappa_{\alpha \alpha'} \rho_1^\alpha(\vec{r}) \rho_2^{\alpha'}(\hat{R} \vec{r}) = \sum_{\alpha \alpha'} \sum_{i\in\rho_1^\alpha} \sum_{j\in\rho_2^{\alpha'}} \kappa_{\alpha\alpha'}\pi^{3/2}\sigma^3 \exp\left(-\frac{(\vec{x}_i-\hat{R}\vec{x}_j)}{4\sigma^2}\right).
\end{equation}
The full kernel may therefore be written as
\begin{equation}
k(\rho_1, \rho_2) = \pi^3 \sigma^6 \sum_{\alpha\alpha'\beta\beta'} \kappa_{\alpha\alpha'} \kappa_{\beta\beta'} \sum\limits_{\substack{i\in\rho_1^\alpha \\ j_\in\rho_2^{\alpha'}}} \sum\limits_{\substack{k \in \rho_1^\beta \\ l \in \rho_2^{\beta'}}} \int d\hat{R} \exp\left(-\frac{(\vec{x}_i - \hat{R}\vec{x}_j)^2}{4 \sigma^2}\right)\exp\left(-\frac{(\vec{x}_k - \hat{R}\vec{x}_l)^2}{4 \sigma^2}\right).
\end{equation}
This kernel captures angular information between at most two atoms in the environment. It is therefore not expected to be more accurate than Glielmo's three-body kernel \cite{glielmo2018efficient}, which directly compares pairs and triplets of distances without performing an integral over SO(3).

Normalizing the kernel and raising to a positive power may allow the model to capture many-body interactions, but the same procedure can be applied to the two- and three-body kernels of Ref.\ \cite{glielmo2018efficient}.


\section{Computing the power spectrum}
We wish to efficiently compute the power spectrum
\begin{equation}
p_{n \nu l} = \sqrt{\frac{8\pi^2}{2l+1}} \sum_m c_{nlm}c^*_{\nu l m},
\end{equation}
where $c_{nlm}$ are expansion coefficients determined by
\begin{equation}
    \rho(\vec{r}) = \sum_{nlm} c_{nlm} g_{nl}(r) Y_{lm}(\hat{r}).
\end{equation}
$\rho(\vec{r})$ is the atomic neighbor density
\begin{equation}
    \rho(\vec{r}) = \sum_i \exp(-\alpha |\vec{r} - \vec{r}_i|^2)
\end{equation}
and $g_{nl}(r)$ are an orthonormal set of radial basis functions
\begin{equation}
    \int dr r^2 g_{nl}(r) g_{n' l}(r) = \delta_{n n'}.
\end{equation}
Note that we allow the radial basis set to depend on $l$.


From \cite{bartok2013representing}, we have the following exact formula for $\rho(\vec{r})$:
\begin{equation}
    \rho(\vec{r}) = \sum_{ilm} c_{lm}^i(r) Y_{lm}(\hat{r})
\end{equation}
where
\begin{equation}
    c_{lm}^i(r) = 4\pi \exp(-\alpha(r^2+r_i^2)) i_l (2\alpha r r_i) Y_{lm}^*(\hat{r}_i).
\end{equation}
Here $i_l(x)$ is the modified spherical Bessel function of the first kind and is related to the modified Bessel function of the first kind $I_l(x)$ according to
\begin{equation}
    i_l(x) = \sqrt{\frac{\pi}{2 x}} I_{l+1/2}(x).
\end{equation}

Let $\phi_{nl}(r)$ be a set of functions and let
\begin{equation}
g_{nl}(r) = \sum_{n'} U_{n n'} \phi_{n'}(r)
\end{equation}
be an orthonormal basis constructed out of $\phi_n(r)$. Then 
\begin{equation}
    c_{nlm} = \sum_{i n'} 4\pi Y^*_{lm}(\hat{r}_i) U_{n n'} I_{n'l}(r_i)
\end{equation}
where
\begin{equation}
    I_{n' l} \equiv \int_0^\infty dr r^2 \exp(-\alpha(r^2 + r_i^2)) i_l(2\alpha r r_i) \phi_{n' l}(r).
\end{equation}
The integral identity \cite{dlmf}
\begin{equation}
\int_0^\infty t^{\nu+1} I_\nu(bt)\exp(-p^2 t^2) dt = \frac{b^\nu}{(2p^2)^{\nu+1}} \exp\left(\frac{b^2}{4p^2}\right)
\end{equation}
suggests that we choose the functions
\begin{equation}
    \phi_{nl}(r) = r^l \exp\left(-\beta_n r^2\right),
\end{equation}
which gives
\begin{equation}
I_{nl}(r_i) = \frac{1}{4} \sqrt{\frac{\pi}{\alpha r_i}} \exp\left(-\frac{ \alpha \beta_n r_i^2}{\alpha + \beta_n}\right) \frac{(\alpha r_i)^{l+1/2}}{(\alpha+\beta_n)^{l+3/2}}.
\end{equation}
Defining
\begin{equation}
a_{nl}(r_i) \equiv \sum_{n'} U_{n n'} I_{n' l}(r_i),
\end{equation}
we may write the expansion coefficients $c_{nlm}$ as
\begin{equation}
c_{nlm} = \sum_i 4\pi a_{nl}(r_i) Y^*_{lm}(\hat{r}_i).
\end{equation}
The power spectrum then takes the form
\begin{equation}
\begin{split}
p_{n\nu l} &= \frac{32 \pi^3 \sqrt{2}}{\sqrt{2l+1}} \sum_{m i i'} a_{nl}(r_i) a_{\nu l}(r_{i'}) Y^*_{lm}(\hat{r}_i) Y_{lm}(\hat{r}_{i'}) \\
&= \frac{32 \pi^3 \sqrt{2}}{\sqrt{2l+1}} \sum_{i i'} a_{nl}(r_i) a_{\nu l}(r_{i'}) \left( \sum_m Y^*_{lm}(\hat{r}_i) Y_{lm}(\hat{r}_{i'}) \right).
\end{split}
\end{equation}
By the addition theorem 
\begin{equation}
\sum_m Y_{lm}(\hat{x}) Y^*_{lm}(\hat{y}) = \frac{2l+1}{4\pi} P_l(\hat{x} \cdot \hat{y}),
\end{equation}
where $P_l$ is the Legendre polynomial of degree $l$, the expression for the power spectrum becomes
\begin{equation}
    \boxed{p_{n\nu l} = 8\pi^2 \sqrt{4l+2} \sum_{i i'} a_{nl}(r_i) a_{\nu l}(r_{i'}) P_l(\hat{r}_i \cdot \hat{r}_{i'})}
\end{equation}

\section{Computing the derivative of the power spectrum}
To predict forces, the derivative of the power spectrum with respect to the Cartesian coordinates of the central atom is needed. Let $\xi_{ki}$ denote the $k^{\text{th}}$ Cartesian component of atom $i$ and $\xi_{k0}$ denote the $k^{\text{th}}$ Cartesian component of the central atom. The derivative of the power spectrum is given by
\begin{equation}
    \boxed{
\begin{split}
\frac{\partial p_{n\nu l}}{\partial \xi_{k0}} = 8 \pi^2 \sqrt{4l + 2} \sum_{ii'} \bigg( &\frac{\partial a_{nl}(r_i)}{\partial r_i} \frac{\partial r_i}{\partial \xi_{k0}} a_{\nu l}(r_{i'}) P_l(\hat{r}_i \cdot \hat{r}_{i'}) + \\
&a_{nl}(r_i) \frac{\partial a_{\nu l}(r_{i'})}{\partial r_{i'}} \frac{\partial r_{i'}}{\partial \xi_{k0}} P_l(\hat{r}_i \cdot \hat{r}_{i'}) + \\
&a_{nl}(r_i) a_{\nu l}(r_{i'}) \frac{\partial P_l(\hat{r}_i \cdot \hat{r}_{i'})}{\partial (\hat{r}_i \cdot \hat{r}_{i'})} \frac{{\partial (\hat{r}_i \cdot \hat{r}_{i'})}}{\partial \xi_{k0}}  \bigg)
\end{split}
    }
\end{equation}
We now show how each of the derivatives appearing in this expression can be efficiently evaluated. First, note that
\begin{equation}
a_{nl}'(r_i) \equiv \frac{\partial a_{nl}}{\partial r_i} = \sum_{n'} U_{n n'} I_{n'l}'(r_i),
\end{equation}
where
\begin{equation}
I_{nl}'(r_i) \equiv \frac{\partial I_{nl}}{\partial r_i} = \frac{\sqrt{\pi}\alpha}{4} \left(\frac{1}{\alpha r_i} \right)^{3/2} \frac{(\alpha r_i)^{l+1/2}}{(\alpha+\beta_n)^{l+5/2}} (-2 r_i^2 \alpha \beta_n + l(\alpha+\beta_n)).
\end{equation}
Next, note that
\begin{equation}
\frac{\partial r_i}{\partial \xi_{k0}} = -\frac{\xi_{ki}}{r_i},
\end{equation}
assuming the central atom is fixed at the origin. The derivative of the Legendre polynomial may be efficiently evaluated using the identity
\begin{equation}
\frac{d P_n(x)}{d x} = \frac{n}{x^2-1}(x P_n(x) - P_{n-1}(x)).
\end{equation}
Finally, it is easy to show that
\begin{equation}
\frac{\partial (\hat{r}_i \cdot \hat{r}_j)}{\partial \xi_0} = \frac{-(\xi_{ki} + \xi_{kj}) r_i r_j + (\hat{r}_i \cdot \hat{r}_j)(r_j^2 \xi_{ki}+r_i^2 \xi_{kj})}{r_i^2 r_j^2},
\end{equation}
again assuming that the central atom is fixed at the origin.

\section{Computing the derivative of the SOAP kernel}

The full SOAP kernel is the dot product of normalized power spectra raised to a positive power. In our approach, the covariance between local energies $\epsilon_i$ is modelled by the SOAP kernel,
\begin{equation}
\langle \epsilon_i \epsilon_j \rangle = K(\hat{p}_i, \hat{p}_j) = \left( \hat{p}_i \cdot \hat{p}_j \right)^{\xi},
\end{equation}
where $\hat{p}_i$ denotes the normalized power spectrum of environment $i$. To model forces, the covariance of the partial derivatives of the local energies is needed:
\begin{equation}
    \begin{split}
\bigg\langle \frac{\partial \epsilon_i}{\partial \xi_{ik}}, \frac{\partial \epsilon_j}{\partial\xi_{jl}}\bigg\rangle &= \frac{\partial^2 \langle \epsilon_i \epsilon_j \rangle}{\partial \xi_{ik} \partial \xi_{jl}}\\
&= \frac{\partial^2 K(\hat{p}_i, \hat{p}_j)}{\partial \xi_{ik} \partial \xi_{jl}} \\
&= \sum_{\alpha \beta} \frac{\partial p_{i\alpha}}{\partial \xi_{ik}} \frac{\partial^2 K(\hat{p}_i, \hat{p}_j)}{\partial p_{i\alpha} \partial p_{j\beta}} \frac{\partial p_{j\beta}}{\partial \xi_{jl}},
    \end{split}
\end{equation}
where $p_{i\alpha}$ is component $\alpha$ of the normalized power spectrum $\hat{p}_i$. The derivatives of the power spectra with respect to Cartesian components may be evaluated using the expressions from Section 2. The remaining term is
\begin{equation}
\begin{split}
\frac{\partial^2 K(\hat{p}_i, \hat{p}_j)}{\partial p_{i\alpha} \partial p_{j\beta}} &= \frac{\partial}{\partial p_{i \alpha}} \left( \xi (\hat{p}_i \cdot \hat{p}_j)^{\xi - 1} p_{i \beta} \right) \\
&= \xi (\xi - 1) (\hat{p}_i \cdot \hat{p}_j)^{\xi - 2} p_{j\alpha} p_{i \beta} + \xi (\hat{p}_i \cdot \hat{p}_j)^{\xi - 1} \delta_{\alpha \beta} \\
&= \xi (\hat{p}_i \cdot \hat{p}_j)^{\xi - 2} \left( (\xi - 1) p_{j\alpha} p_{i\beta} + (\hat{p}_i \cdot \hat{p}_j) \delta_{\alpha \beta} \right).
\end{split}
\end{equation}
Notice that if $\xi = 1$, the double sum collapses into a single sum. Raising the SOAP kernel to a power $\xi > 1$ therefore comes at a significant computational cost. In \cite{szlachta2014accuracy}, it is claimed that the role of $\xi$ is to sharpen the selectivity of the similarity measure. However, it seems that the hyperparameter $\alpha$ plays the exact same role. To see this, notice that in the limit of infinite $\alpha$, the atomic density becomes a sum of Dirac delta functions, and the covariance is nonzero only if the atomic environments are identical. In the limit $\alpha \rightarrow 0$, the atomic density becomes a constant, and all atomic densities are treated the same (up to a constant). Tuning $\alpha$ properly may make the $\xi$ hyperparameter unnecessary.


\section{Review of the single-component SOAP kernel}

In the SOAP scheme, an atomic environment is represented by an atomic neighbor density function $\rho(\vec{r})$, where
\begin{equation}
\rho(\vec{r}) = \sum_i \exp\left(-\alpha | \vec{r} - \vec{r}_i|^2 \right),
\end{equation}
$\alpha$ is a hyperparameter, and $\vec{r}_i$ is the position of atom $i$ in the environment of the central atom. The single-component SOAP kernel between two atomic neighbor densities $\rho$ and $\rho'$ is defined by Eq.\ (36) of \cite{bartok2013representing} as:

\begin{equation}
    K(\rho, \rho') = \left(\frac{k (\rho, \rho')}{\sqrt{k(\rho, \rho) k(\rho', \rho')}} \right) ^{\xi},
\end{equation}
where $k(\rho, \rho')$ is a rotationally invariant symmetry kernel given by
\begin{equation}
k(\rho, \rho') = \int d\hat{R} \left| \int \rho(\vec{r}) \rho'(\hat{R}\vec{r}) d\vec{r} \right| ^n,
\end{equation}
and $\hat{R}$ is a 3-D rotation. The outer integral is over all rotations and ensures that the kernel is rotationally symmetric.

In Section IV(B) of \cite{bartok2013representing}, it is shown that $k(\rho, \rho')$ is equal to the dot product of the ``rotational power spectra'' of densities $\rho$ and $\rho'$, which is efficient to compute. I will review this derivation here, since a similar argument can be applied to multi-component systems.

Letting $g_n(r)$ be an orthonormal set of radial basis functions, so that
\begin{equation}
\int dr r^2 g_n(r) g_{n'}(r) = \delta_{n n'},
\end{equation}
an atomic neighbor density $\rho(\vec{r})$ may be expanded into spherical harmonics as
\begin{equation}
\rho(\vec{r}) = \sum_{n l m} c_{n l m} g_n(r) Y_{l m} (\vec{r}).
\end{equation}
We first compute that
\begin{equation} \label{S_eq}
\begin{split}
   S(\rho, \hat{R} \rho') &\equiv \int d\vec{r} \rho(\vec{r}) \rho'(\hat{R} \vec{r}) \\
   &= \int d\vec{r} \left(\sum_{n l m} c_{n l m}^* g_n(r)Y_{l m}^*(\hat{r}) \right)\left(\sum_{n' l' m'} c'_{n' l' m'} g_{n'}(r)Y_{l' m'}(\hat{R} \vec{r}) \right) \\
   &= \sum_{n l m n' l' m'} c^*_{n l m} c'_{n' l' m'} \int dr r^2 g_n(r) g_{n'}(r) \int d\hat{r} Y^*_{l m}(\hat{r}) Y_{l' m'}(\hat{R} \vec{r}) \\
   &= \sum_{n l m l' m'} c^*_{n l m} c'_{n l' m'} \int d\hat{r} Y^*_{l m}(\hat{r}) Y_{l' m'} (\hat{R} \vec{r}).
\end{split}
\end{equation}
Note that
\begin{equation}
    \begin{split}
Y_{l' m'}(\vec{R} \hat{r}) &= \braket{\vec{r} | \hat{R}^\dagger | Y_{l' m'}} \\
&= \sum_{m''} \braket{\vec{r} | Y_{l' m''}} \braket{Y_{l' m''} | \hat{R}^\dagger | Y_{l' m'}} \\
&= \sum_{m''} \braket{\vec{r} | Y_{l' m''}} \braket{Y_{l' m'} | \hat{R} | Y_{l' m''}}^* \\
&= \sum_{m''} Y_{l' m''}(\hat{r}) D^*(\hat{R})_{m' m''}^{l'},
    \end{split}
\end{equation}
where the Wigner function $D(\hat{R})_{m m'}^{l}$ is defined as
\begin{equation}
D(\hat{R})_{m m'}^{l} = \braket{Y_{l m} | \hat{R} | Y_{l m'}}.
\end{equation}
Plugging into Eq.\ (\ref{S_eq}) gives
\begin{equation}
    \begin{split}
S(\rho, \hat{R} \rho') &= \sum_{n l m l' m' m''} c^*_{n l m} c'_{n l' m'} D^*(\hat{R})^{l'}_{m' m''} \int d\hat{r} Y^*_{l m }(\hat{r}) Y_{l' m''}(\hat{r}) \\
&= \sum_{n l m m'} c^*_{n l m} c'_{n l m'} D^*(\hat{R})^l_{m' m}.
    \end{split}
\end{equation}
The symmetry kernel $k(\rho, \rho')$ with $n = 2$ then takes the form
\begin{equation}
    \begin{split}
k(\rho, \rho') &= \int d\hat{R} S(\rho, \hat{R}\rho') S^*(\rho, \hat{R} \rho') \\
&= \sum_{n l m m' n' \lambda \mu \mu'} c^*_{n l m} c'_{n l m'} c_{n' \lambda \mu} (c'_{n' \lambda \mu'})^* \int d\hat{R} D^*(\hat{R})^l_{m' m} D(\hat{R})^\lambda_{\mu' \mu} \\
&= \sum_{n l m m' n'} \frac{8 \pi^2}{2l+1} c^*_{n l m} c'_{n l m'} c_{n' l m} (c'_{n' l m'})^* \\
&= \sum_{n n' l} \frac{8\pi^2}{2l+1} \left(\sum_{m} c^*_{n l m} c_{n' l m} \right) \left(\sum_{m'} c'_{n l m'} (c'_{n' l m'})^* \right) \\
&= \sum_{n n' l} \frac{8\pi^2}{2l+1} p_{n' n l} p'_{n n' l},
    \end{split}
\end{equation}
where $p_{n n' l} \equiv \sum_{m} c_{n l m} c_{n' l m}^*$ is the rotational power spectrum. The power spectrum of a real function is real, so $p_{n' n l} = p_{n n' l}$, and $k(\rho, \rho')$ may be expressed as the dot product of the power spectra $\vec{p}$ and $\vec{p'}$:
\begin{equation}
k(\rho, \rho') = \vec{p} \cdot \vec{p'}.
\end{equation}
The full SOAP kernel $K(\rho, \rho')$ is the dot product of normalized power spectra raised to a positive power:
\begin{equation}
K(\rho, \rho') = \left( \frac{\vec{p}}{|\vec{p}|} \cdot \frac{\vec{p'}}{|\vec{p}'|} \right)^{\xi}.
\end{equation}

% \subsection{Computing the SOAP kernel}
% The key step in compuing $K(\rho, \rho')$ is determining the coefficients $c_{n l m}(r, \{\vec{r}_i\})$, which depend on the radial distance $r$, the positions $\vec{r}_i$ of the environment atoms, and the radial basis set.

\section{Review of the Multi-component SOAP kernel}
We now review the multi-species version of the SOAP kernel proposed in \cite{de2016comparing}. Consider a two-component system. Let $\rho$ and $\rho'$ denote two atomic environments, with $\rho_A, \rho_B$ the atomic neighbor densities of atom types $A$ and $B$ in environment $\rho$, and $\rho'_A, \rho'_B$ the densities of $A$ and $B$ in environment $\rho'$. We define the MOP kernel between $\rho$ and $\rho'$ by direct analogy to the SOAP kernel:
\begin{equation}
    K_{\text{MOP}}(\rho, \rho') = \left(\frac{k_{\text{MOP}} (\rho, \rho')}{\sqrt{k_{\text{MOP}}(\rho, \rho) k_{\text{MOP}}(\rho', \rho')}} \right) ^{\xi},
\end{equation}
where $k_{\text{MOP}}(\rho, \rho')$ is a rotationally invariant symmetry kernel given by
\begin{equation}
    \begin{split}
    k_{\text{MOP}}(\rho, \rho') &= \int d\hat{R} \left| \int \left( \rho_A(\vec{r}) \rho'_A(\hat{R}\vec{r}) + \rho_B(\vec{r}) \rho'_B(\hat{R}\vec{r}) \right) d\vec{r} \right| ^2 \\
    &= \int d\hat{R} \left| \int \rho_A(\vec{r}) \rho'_A(\hat{R}\vec{r}) d\vec{r} \right|^2 + \int d\hat{R} \left| \int \rho_B(\vec{r}) \rho'_B(\hat{R}\vec{r}) d\vec{r} \right|^2 \\
    &+ \int d \hat{R} \left( \int d\vec{r} \rho_A(\vec{r}) \rho'_A(\hat{R} \vec{r}) \right) \left( \int d\vec{r} \rho_B(\vec{r}) \rho'_B(\hat{R} \vec{r}) \right)^* \\
    &+ \int d \hat{R} \left( \int d\vec{r} \rho_A(\vec{r}) \rho'_A(\hat{R} \vec{r}) \right)^* \left( \int d\vec{r} \rho_B(\vec{r}) \rho'_B(\hat{R} \vec{r}) \right) \\
    &= k(\rho_A, \rho'_A) + k(\rho_B, \rho'_B) + k_{\text{cross}}(\rho, \rho') + k_{\text{cross}}^*(\rho, \rho'),
    \end{split}
\end{equation}
which is the sum of the SOAP kernels between the two density types plus a cross term $k_{\text{cross}}$, which we now analyze in detail.

From the derivation in part 1, we have that
\begin{equation}
\int d\vec{r} \rho_A(\vec{r}) \rho'_A(\hat{R}\hat{r}) = \sum_{n l m m'} (c_{n l m}^A)^* c_{n l m}^{'A} D^* (\hat{R})^l_{m' m},
\end{equation}

\begin{equation}
\int d\vec{r} \rho_B(\vec{r}) \rho'_B(\hat{R}\hat{r}) = \sum_{n' \lambda \mu \mu'} (c_{n' \lambda \mu}^B)^* c_{n' \lambda \mu}^{'B} D^* (\hat{R})^l_{\mu' \mu},
\end{equation}
and thus
\begin{equation}
    \begin{split}
k_{\text{cross}}(\rho, \rho') &= \int d\hat{R} S(\rho_A, \hat{R} \rho'_A) S^*(\rho_B, \hat{R}\rho'_B) \\
&= \sum_{n l m m' n' \lambda \mu \mu'} (c_{n l m}^A)^* c_{n l m'}^{'A} c_{n' \lambda \mu}^B (c_{n' \lambda \mu'}^{'B})^* \int d\hat{R} D^*(\hat{R})_{m' m}^l D(\hat{R})_{\mu' \mu}^\lambda \\
&= \sum_{n l m m' n'} \frac{8\pi^2}{2l + 1} (c_{n l m}^A)^* c_{n l m'}^{'A} c_{n' l m}^B (c_{n' l m'}^{'B})^* \\
&= \sum_{n n' l} \frac{8\pi^2}{2l+1} \left( \sum_{m} (c_{n l m}^A)^* c_{n' l m}^B \right) \left( \sum_{m} c_{n l m'}^{'A} (c_{n' l m'}^{'B})^* \right) \\
&= \vec{p}_{AB} \cdot \vec{p}'_{AB},
    \end{split}
\end{equation}
which is the dot product of the cross power spectra of the two environments. The symmetric MOP kernel $k_{\text{MOP}}(\rho, \rho')$ before normalization therefore takes the form
\begin{equation}
k_{\text{MOP}}(\rho, \rho') = \vec{p}_A \cdot \vec{p}'_A + \vec{p}_B \cdot \vec{p}'_B + 2 \vec{p}_{AB} \cdot \vec{p}'_{AB}.
\end{equation}
Notice that if one of the densities vanishes, e.g.\ $\vec{\rho}_A = 0$, the MOP kernel reduces to the SOAP kernel. MOP may therefore be viewed as a generalization of SOAP that is suitable for multicomponent systems.

\bibliography{multi_soap.bib}
\end{document}