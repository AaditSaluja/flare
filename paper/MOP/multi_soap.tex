\documentclass[%
% reprint,
%superscriptaddress,
%groupedaddress,
%unsortedaddress,
%runinaddress,
%frontmatterverbose, 
preprint,
%showpacs,preprintnumbers,
%nofootinbib,
%nobibnotes,
%bibnotes,
amsmath,amssymb,
aps,
%pra,
%prb,
%rmp,
%prstab,
%prstper,
%floatfix,
]{revtex4-1}

\usepackage{braket}
\usepackage{graphicx}% Include figure files
% \usepackage{dcolumn}% Align table columns on decimal point
\usepackage{bm}% bold math
%\usepackage{hyperref}% add hypertext capabilities
%\usepackage[mathlines]{lineno}% Enable numbering of text and display math
%\linenumbers\relax % Commence numbering lines

%\usepackage[showframe,%Uncomment any one of the following lines to test 
%%scale=0.7, marginratio={1:1, 2:3}, ignoreall,% default settings
%%text={7in,10in},centering,
%%margin=1.5in,
%%total={6.5in,8.75in}, top=1.2in, left=0.9in, includefoot,
%%height=10in,a5paper,hmargin={3cm,0.8in},
%]{geometry}

\begin{document}

\title{Some notes on extending the SOAP kernel to multi-component systems}

\author{Jonathan Vandermause}
% \affiliation{$^{1}$John A. Paulson School of Engineering and Applied
% Sciences, Harvard University, Cambridge, Massachusetts 02138, USA \\
% $^{2}$Department of Physics, Harvard University, Cambridge, Massachusetts
% 02138, USA \\
% $^{3}$Center for Computational Engineering, Massachusetts Institute of
% Technology, Cambridge, Massachusetts 02139, USA \\
% $^{4}$ Department of Mechanical Engineering, Massachusetts Institute of Technology, Cambridge, Massachusetts 02139, USA}

\date{\today}

% \begin{abstract}
% \end{abstract}

\maketitle

\section{Review of the single-component SOAP kernel}

\subsection{SOAP as a dot product of normalized rotational power spectra}

In the SOAP scheme, an atomic environment is represented by an atomic neighbor density function $\rho(\vec{r})$, where
\begin{equation}
\rho(\vec{r}) = \sum_i \exp\left(-\alpha | \vec{r} - \vec{r}_i|^2 \right),
\end{equation}
$\alpha$ is a hyperparameter, and $\vec{r}_i$ is the position of atom $i$ in the environment of the central atom. The single-component SOAP kernel between two atomic neighbor densities $\rho$ and $\rho'$ is defined by Eq.\ (36) of \cite{bartok2013representing} as:

\begin{equation}
    K(\rho, \rho') = \left(\frac{k (\rho, \rho')}{\sqrt{k(\rho, \rho) k(\rho', \rho')}} \right) ^{\xi},
\end{equation}
where $k(\rho, \rho')$ is a rotationally invariant symmetry kernel given by
\begin{equation}
k(\rho, \rho') = \int d\hat{R} \left| \int \rho(\vec{r}) \rho'(\hat{R}\vec{r}) d\vec{r} \right| ^n,
\end{equation}
and $\hat{R}$ is a 3-D rotation. The integral is over all rotations and ensures that the kernel is rotationally symmetric.

In Section IV(B) of \cite{bartok2013representing}, it is shown that $k(\rho, \rho')$ is equal to the dot product of the ``rotational power spectra'' of densities $\rho$ and $\rho'$, which is efficient to compute. I will review this derivation here, since a similar argument can be applied to multi-component systems.

Letting $g_n(r)$ be an orthonormal set of radial basis functions, so that
\begin{equation}
\int dr r^2 g_n(r) g_{n'}(r) = \delta_{n n'},
\end{equation}
an atomic neighbor density $\rho(\vec{r})$ may be expanded into spherical harmonics as
\begin{equation}
\rho(\vec{r}) = \sum_{n l m} c_{n l m} g_n(r) Y_{l m} (\vec{r}).
\end{equation}
We first compute that
\begin{equation} \label{S_eq}
\begin{split}
   S(\rho, \hat{R} \rho') &\equiv \int d\vec{r} \rho(\vec{r}) \rho'(\hat{R} \vec{r}) \\
   &= \int d\vec{r} \left(\sum_{n l m} c_{n l m}^* g_n(r)Y_{l m}^*(\hat{r}) \right)\left(\sum_{n' l' m'} c'_{n' l' m'} g_{n'}(r)Y_{l' m'}(\hat{R} \vec{r}) \right) \\
   &= \sum_{n l m n' l' m'} c^*_{n l m} c'_{n' l' m'} \int dr g_n(r) g_{n'}(r) \int d\hat{r} Y^*_{l m}(\hat{r}) Y_{l' m'}(\hat{R} \vec{r}) \\
   &= \sum_{n l m l' m'} c^*_{n l m} c'_{n l' m'} \int d\hat{r} Y^*_{l m}(\hat{r}) Y_{l' m'} (\hat{R} \vec{r}).
\end{split}
\end{equation}
Note that
\begin{equation}
    \begin{split}
Y_{l' m'}(\vec{R} \hat{r}) &= \braket{\vec{r} | \hat{R}^\dagger | Y_{l' m'}} \\
&= \sum_{m''} \braket{\vec{r} | Y_{l' m''}} \braket{Y_{l' m''} | \hat{R}^\dagger | Y_{l' m'}} \\
&= \sum_{m''} \braket{\vec{r} | Y_{l' m''}} \braket{Y_{l' m'} | \hat{R} | Y_{l' m''}}^* \\
&= \sum_{m''} Y_{l' m''}(\hat{r}) D^*(\hat{R})_{m' m''}^{l'},
    \end{split}
\end{equation}
where the Wigner function $D(\hat{R})_{m m'}^{l}$ is defined as
\begin{equation}
D(\hat{R})_{m m'}^{l} = \braket{Y_{l m} | \hat{R} | Y_{l m'}}.
\end{equation}
Plugging into Eq.\ (\ref{S_eq}) gives
\begin{equation}
    \begin{split}
S(\rho, \hat{R} \rho') &= \sum_{n l m l' m' m''} c^*_{n l m} c'_{n l' m'} D^*(\hat{R})^{l'}_{m' m''} \int d\hat{r} Y^*_{l m }(\hat{r}) Y_{l' m''}(\hat{r}) \\
&= \sum_{n l m m'} c^*_{n l m} c'_{n l m'} D^*(\hat{R})^l_{m' m}.
    \end{split}
\end{equation}
The symmetry kernel $k(\rho, \rho')$ with $n = 2$ then takes the form
\begin{equation}
    \begin{split}
k(\rho, \rho') &= \int d\hat{R} S(\rho, \hat{R}\rho') S^*(\rho, \hat{R} \rho') \\
&= \sum_{n l m m' n' \lambda \mu \mu'} c^*_{n l m} c'_{n l m'} c_{n' \lambda \mu} (c'_{n' \lambda \mu'})^* \int d\hat{R} D^*(\hat{R})^l_{m' m} D(\hat{R})^\lambda_{\mu' \mu} \\
&= \sum_{n l m m' n'} c^*_{n l m} c'_{n l m'} c_{n' l m} (c'_{n' l m'})^* \\
&= \sum_{n n' l} \left(\sum_{m} c^*_{n l m} c_{n' l m} \right) \left(\sum_{m'} c'_{n l m'} (c'_{n' l m'})^* \right) \\
&= \sum_{n n' l} p_{n' n l} p'_{n n' l}.
    \end{split}
\end{equation}
Because the atomic densities $\rho$ and $\rho'$ are real-valued functions, $p_{n' n l} = p_{n n' l}$, so $k(\rho, \rho')$ may be expressed as the dot product of the power spectra $\vec{p}$ and $\vec{p'}$:
\begin{equation}
k(\rho, \rho') = \vec{p} \cdot \vec{p'}.
\end{equation}
The full SOAP kernel $K(\rho, \rho')$ is
\begin{equation}
K(\rho, \rho') = \left( \frac{\vec{p}}{|\vec{p}|} \cdot \frac{\vec{p'}}{|\vec{p}'|} \right)^{\xi}.
\end{equation}

\subsection{Computing the SOAP kernel}
The key step in compuing $K(\rho, \rho')$ is determining the coefficients $c_{n l m}(r, \{\vec{r}_i\})$, which depend on the radial distance $r$, the positions $\vec{r}_i$ of the environment atoms, and the radial basis set.
\bibliography{multi_soap.bib}
\end{document}